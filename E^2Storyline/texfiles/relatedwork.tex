%% The ``\maketitle'' command must be the first command after the
%% ``\begin{document}'' command. It prepares and prints the title block.
	
%% the only exception to this rule is the \firstsection command
%% \section{Introduction} %for journal use above \firstsection{..} instead
	
	
\section{RELATED WORK} %2

\subsection{Visualization of temporal events}%2.1 时间性事件可视化
%时间性事件可视化是一种基于时间的用来说明事件随时间顺序的可视化技术。由于可视化技术的多样性和广泛可用性,对时间性事件序列数据可视化的总结工作也有不少好的成果。Guo回顾了最先进的视觉分析方法,并提出了新的设计空间对它们进行描述、分类。chao通过对可视化中讲故事的文献进行调查,对其中常见和重要的元素进行概述,并描述了这一领域的挑战。
\noindent Temporal event visualization is a time-based visualization technique used to illustrate the events in chronological order. Due to the diversity and widespread availability of visualization techniques, summary work on the visualization of temporal event series data has yielded several favorable results. Guo~\cite{guo_survey_2021} reviewed the state-of-the-art visual analysis methods and proposes a new design space to describe and classify them. Through a survey of the literature on storytelling in visualization, Tong~\cite{tong_storytelling_2018} provided an overview of common and important elements and describes the challenges in this field.

%目前构建时间事件的可视化技术时采用的时间轴的表现形式也有很多。例如:Lifeline,等将时间轴用直线表示。该工作中事件按照其发生的时间在时间线上进行显示。也有一些工作将时间轴用圆形或者弧线表示。StoryPrint,用圆形时间轴的同心圆环对电影中的场景、角色进行可视化表现,并进行比较分析。时间轴形状对任务的性能也存在一定的影响。Sara基于用户任务和数据类型创建有效的时间轴可视化的设计指南。
Many visualization techniques are used to construct temporal events, and the timeline can be represented in many forms. Lifeline~\cite{plaisant_lifelines_2003}, for example, represented the timeline as a straight line, in which events are displayed on the timeline according to their time of occurrence. Others work to represent the timeline as a circle or an arc. StoryPrint~\cite{watson_storyprint_2019}, a visual representation of scenes and characters in a movie with concentric rings in a circular timeline, and comparative analysis. The shape of the timeline also has an implication on the performance of the task. Design guidelines for creating effective timeline visualizations based on user tasks and data types have been proposed~\cite{di_bartolomeo_evaluating_2020}.

%通常用于分析事件序列的方法有聚合,细节展示等。常见的聚合形式有桑吉图,冰柱图,以及转移矩阵,它们展示的事件都有相同的时间顺序。尽管它们可以可视化所有事件的排列,但可伸缩性差,不能展示细节层次。通过构建故事概述,或者对事件、时间进行选择或者定义一些规则可以缩减事件规模。用户再借助一些交互手段,可以对细节进行探索分析。为了弥合视觉分析和故事叙述之间的差距,chen提出了一个总体框架,将数据分析和结果展示通过故事合成联系在一起。sequence braiding利用分层有向无环网络对事件时间序列和属性进行总体的可视化。
Typical methods used to analyze event sequences are aggregation, detail presentation, and so on. Some of the common forms of aggregation are Sankey diagrams~\cite{wongsuphasawat_exploring_2012, riehmann_interactive_2005}, icicle diagrams~\cite{monroe_temporal_2013, wongsuphasawat_lifeflow_2011}, and transfer matrices~\cite{yi_timematrix_2010, zhao_matrixwave_2015, bach_visualizing_2014}, which show events in the same chronological order. Despite the fact that they can visualize the arrangement of all events, their scalability is poor and cannot show the level of detail. By building an overview of the story~\cite{perer_frequence_2014, liu_coreflow_2017}, or by selective of events and times~\cite{monroe_temporal_2013, baumgartl_search_2020, guo_visual_2018} or by defining some rules~\cite{zgraggen_s_2015, cappers_exploring_2017} can be reduced in size. The user can then explore and analyze the details with the help of some interactions~\cite{magallanes_sequen-c_2021, pena-araya_hyperstorylines_2022}. To bridge the gap between visual analysis and storytelling, Chen~\cite{chen_supporting_2018} proposed an overarching framework that links data analysis and presentation of results through story synthesis. Sequence braiding~\cite{di_bartolomeo_s_2020} utilized hierarchical directed acyclic networks for the overall visualization of event time series and attributes.

\subsection{Storyline visualization} %2.2故事线可视化
%故事线可视化作为时间序列可视化中最常用的技术之一,也常常出现在一些工作中。Ogawa在其工作中,提出了一种可视化软件项目演进中开发人员之间交互的技术,可以在展示更多细节的同时,保持对美学和表现的关注。tan提出了一套设计考虑因素,用来产生审美愉悦和易读的故事线可视化。iStoryline通过研究用户如何将叙事转换成手绘故事线,提出了设计空间,并开发了一个创作工具,以求做到手绘故事线和自动布局之间的平衡。PlotThread基于一个强化学习框架来训练人工智能,从而帮助用户探索设计空间,生成优化的故事情节。MeetingVis针对会议数据,提出了一种基于视觉叙事的会议摘要方法。
\noindent Storyline visualization, as one of the most commonly used techniques in time series visualization, is also often appeared in some works. Ogawa~\cite{ogawa_software_2010} proposed a technique for visualizing interactions between developers in the evolution of software projects that can show more detail while maintaining a focus on aesthetics and presentation. Tanahashi~\cite{tanahashi_design_2012} proposed a set of design considerations for generating aesthetically pleasing and easy-to-read storyline visualizations. iStoryline~\cite{tang_istoryline_2018} proposed a design space by studying how users translate narratives into hand-drawn storylines, and iStoryline developed an authoring tool to achieve a balance between hand-drawn storylines and automatic layout. Based on a reinforcement learning framework to train artificial intelligence, PlotThread~\cite{tang_plotthread_2020} helpd users explore the design space and generate optimized storylines. MeetingVis~\cite{shi_meetingvis_2018} proposed a visual narrative-based meeting summary method for meeting data.

%布局优化在可视化中是非常重要的一环。简洁优美、可读性强的布局是该领域研究人员一直追求的。为此,一些设计原则和优化目标被提出,在多个类型的数据中是通用的。在故事线可视化中常见的设计原则有:1、同一集合的(有关系的)的线条应该彼此靠近;2、线条需要保持直到所在的集合(关系)改变。由此提出的设计目标有:1、减少线摆动;2、减少线交叉;3、较少空白区域。Martain将线交叉最小化问题建模成一个带有树约束的多层最小化问题。MetroSet将集合用地铁图的方式进行可视化,并作了布局优化。Martin使用混合整数规划对地铁图布局和标记进行可视化。EvoRiver在减少空白区域做了部分工作。
Layout optimization is a very important aspect in visualization. Clean, elegant and reader-friendly layouts are always pursued by researchers. To this end, a number of design principles and optimization goals are proposed that are general across multiple types of data~\cite{tanahashi_design_2012, liu_storyflow_2013, tanahashi_efficient_2015}. The design principles in storyline visualization are: \ding{182} Lines from the same set (with relationships) should be close to each other; \ding{183} Lines need to be maintained until the set (relationship) in which they are located changes. The resulting design goals are: \ding{182} less line wiggling; \ding{183} less line crossings; and \ding{184} less white space. Martain~\cite{gronemann_crossing_2016} modeled the line crossing minimization problem as a multilayer minimization problem with tree constraints. MetroSet~\cite{jacobsen_metrosets_2020} visualized sets in terms of Metro maps with layout optimization. Martin~\cite{nollenburg_drawing_2010} used mixed integer programming to visualize the Metro map layout and labeling. EvoRiver~\cite{sun_evoriver_2014} did some work on reducing white space.

%隐喻是有效性和美观性结合的产物,可以在兼顾美观的同时,传达更多的信息。在故事线可视化中,用不同的颜色绘制背景区域可以表达层次信息。SplitStreams用了新颖的视觉隐喻,提供了随时间变化的层次结构的静态概述。除此之外,视觉隐喻的形式也可以采用著名的艺术作品。
Metaphors are the product of combining validity and aesthetics, which can convey more information while taking into account esthetics. In storyline visualization, drawing background areas with different colors expresses hierarchical information~\cite{liu_storyflow_2013, jacobsen_metrosets_2020, reda_visualizing_2011}. SplitStreams~\cite{bolte_splitstreams_2020} used a novel visual metaphor to provide a static overview of the hierarchy over time. Beyond that, visual metaphors can also take the form of prominent artworks~\cite{zhang_visual_2022}.
